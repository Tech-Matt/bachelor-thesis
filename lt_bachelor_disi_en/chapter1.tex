\chapter{What even is a Time to Digital Converter?}
\label{cha:tdc-intro}
The principal objective of this thesis project was the design of a \textbf{time to digital converter}. But what is that? Why and where is it used for? This chapter provides a basic understanding of the researched device and of the ecosystem where it is generally used. Also, from now on, to make the reading easier, I will refer to the device as the TDC.
 



\section{SPADs}
\label{sec:tdc-spads}
Rather unintuitively, the best method to introduce what a TDC is to first speak about one of the main areas where it can be of need. \\ \\
A Single Photon Avalanche Diode (in short - SPAD) is based on a semiconductor PN junction that can be triggered by different kind of radiations, such as the visible light (such diodes are also referred as photodiodes). In particular circuits, this device can be set up so that when a single photon hits the diode, a so called \textbf{avalanche} is formed. The diode is set close to an equilibrium position so that when the photons arrive a chain effect is formed. The diode is able to conduct a large amount of current which can then be sensed by a \textbf{front-end} circuit.
\\ \\
SPADs have many applications, such as LIDARs, low light imaging, PET scanning, fluorescence microscopy and many more.


\section{The TDC}
\label{sec:tdc-introduction}
In all the applications mentioned above there is, often, the need to measure the \textbf{time arrival} or \textbf{time of flight} of photons. For example, in a LIDAR, a photon is sent toward an object, while the TDC starts counting. When the photon is then reflected back to the device we can have a measure of how long the photon has flown (\textbf{Time of Flight}). This measure is useful to reconstruct the distance of an object, and thus, with proper iterations, an image of a space.



\section{Applications of SPADs and TDCs}
\label{sec:tdc-examples}
- Examples
bla \\ bla \\ bla

\section{An Overview of TDC Designs}

\section{The need for a low cost FPGA based TDC}



